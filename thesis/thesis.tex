\documentclass[chaparabic,biol,ms,12pt,oneandhalf,threejury]{metu}
\usepackage{appendix}
\usepackage{longtable}
\usepackage[pdftex]{hyperref}
\usepackage[all]{hypcap}
\usepackage{todonotes}
\usepackage{graphicx}
\usepackage{lscape}
\usepackage{ltablex}
\graphicspath{ {./images/} }
\usepackage[figuresright]{rotating}
\usepackage{xy} 
\usepackage{booktabs}
\usepackage{pifont}
\usepackage{color}
\usepackage{listings}
\usepackage{pdfpages}
\usepackage{array}
\usepackage{algorithm}
\usepackage{algorithmic}
\usepackage{float}
\usepackage{caption}
\usepackage{lastpage}
\usepackage{afterpage}
\usepackage{lipsum}
\usepackage{adjustbox}
\usepackage{rotating}

% \usepackage{graphicx}
\usepackage{amsmath,amssymb} % define this before the line numbering.
% \usepackage{ruler}
\usepackage{color}
% \usepackage{cite}
% \usepackage[utf8x]{inputenc}
% \usepackage{footnote}
% \makesavenoteenv{tabular}
% \makesavenoteenv{table}

\renewcommand{\sectionautorefname}{\S}
\renewcommand{\subsectionautorefname}{\S}

\newcommand{\norm}[1]{\left\lVert#1\right\rVert}

\captionsetup{belowskip=12pt,aboveskip=8pt}
\newcommand{\tab}{\hspace*{2em}}
\DeclareGraphicsExtensions{.pdf,.png,.jpg}


\usepackage{amsmath}
\usepackage{siunitx}
\usepackage{textcomp}
\usepackage{subcaption}


\usepackage{tikz}
\usepackage{mathtools}
% \usepackage{rotating}
%\PassOptionsToPackage{figuresright}{rotating}

\DeclarePairedDelimiter\ceil{\lceil}{\rceil}
\DeclarePairedDelimiter\floor{\lfloor}{\rfloor}


\newcommand{\EA}[1]{\textcolor{red}{[EA: #1]}}

% Name and Surname
\author{Ulaş Işıldak}
% Thesis Title English and Turkish
\title{Meta-Analysis of Gene Expression Heterogeneity in Brain Development and Aging}
\turkishtitle{Beyin Gelişimi ve Yaşlanmasında Gen Anlatımı Heterojenliğinin Meta-Analizi}

\date{August 2022}
 
% prof : Prof. Dr.
% assocprof : Assoc. Prof. Dr.
% assistprof : Assist. Prof. Dr.
% dr : Dr.
%
% Director of Institute
\director[prof]{Halil Kalıpçılar}
% Head of Department
\headofdept[prof]{Ayşe Gül Gözen}
%
% Supervisor : English and Turkish
\supervisor[prof]{Mehmet Somel}
% \turkishsupervisor{  } %if you will hard-code the academic title
%
% Affiliation of Supervisor in English and possibly in Turkish
\departmentofsupervisor{Biology Dept., METU}

%\cosupervisor[dr]{Itır Önal Ertuğrul}
%\departmentofcosupervisor{Robotics Institute, Carnegie Mellon University}
%
% Committee Members
% In general members are sorted according to their academic titles
%
% Proffesors (1)
% Associate Professors (2)
% Assistant Professors (3)
% Other (4)
% 
% IMPORTANT:  All affiliatons should fit in a single line
% If affiliation line is broken into two lines you should shorten the affiliation by using 
% abbrevations or any other means
%
% First committee member should be the chair of examining committee
% Typically the chair is one of the highest ranked committee members
% Ask your supervisor if you are not sure
\committeememberi[assistprof]{Name Surname}
\affiliationi{Biology, METU}
% Second committee member is always your supervisor
\committeememberii[prof]{Mehmet Somel}
\affiliationii{Biology, METU}
% Third member of committee
\committeememberiii[assistprof]{Name Surname}
\affiliationiii{Computer Engineering, Bilkent University}
% IMPORTANT: If you are Ph.D. student your co-supervisor can not be in your 
% examination committee.

% \def\@proftitlename{Prof. Dr.}\def\@tproftitlename{Prof. Dr.}
% \def\@assocproftitlename{Assoc. Prof. Dr.}\def\@tassocproftitlename{Doç. Dr.}
% \def\@assistproftitlename{Assist. Prof. Dr.}\def\@tassistproftitlename{Yrd. Doç. Dr.}
% \def\@drtitlename{Dr.}\def\@tdrtitlename{Dr.}

% Fourth committee member
% \committeememberiv[assistprof]{Gülşah Tümüklü Özyer}
% \affiliationiv{Computer Engineering, Atatürk University}
% Fifth committee member
% \committeememberv[assistprof]{Jüri}
% \affiliationv{JüriBölüm, Ankara University}
%
% Keywords : English & Turkish, Comma seperated
\keywords{aging, development, gene expression, heterogeneity, human, brain}
\anahtarklm{yaşlanma, gelişim, gen anlatımı, heterojenlik, insan, beyin}
%
% Abstract in English
%
\abstract{
    Aging is a complex process associated with the accumulation of stochastic genetic and epigenetic alterations,
    possibly leading to functional decline and increased risk for disease and death.
    Although some previous studies investigating age-related expression changes demonstrated a tendency towards increased inter-individual heterogeneity,
    whether it is a function of time that starts at the beginning of life, 
    or its functional consequences and regulations have not been systematically studied.
    In this study, I addressed these questions by the meta-analysis of 19 microarray age-series datasets, comprising 17 brain regions of 298 individuals.
    Investigating the age-related gene expression heterogeneity changes, 
    I found that there is a significant shift towards increased heterogeneity consistency during aging (20 to 98 years of age)
    compared to the post-natal development period (0 to 20 years of age),
    suggesting that increased heterogeneity is not only driven by the general effect of time, 
    but rather it is a specific effect of the aging period.
    Moreover, the genes that become more heterogeneous consistently across all aging datasets were
    found to be associated with biological processes and pathways that are related to neuronal function (i.e., axon guidance, postsynaptic specialization)
    and longevity (i.e., autophagy, mTOR signaling). 
    Gene set enrichment analysis for transcriptional regulators (i.e., miRNAs and transcription factors) further revealed that
    increased heterogeneity is not only associated with specific regulators but also there is a positive correlation between the number of regulators and 
    consistent changes in heterogeneity, indicating the possible role of transcriptional regulators in the underlying mechanism.
    
    Overall, the results presented here demonstrate that increased inter-individual expression heterogeneity is a general characteristic of the aging human brain,
    which is associated with multiple lifespans and disease-related pathways and processes,
    suggesting that increased heterogeneity may contribute to the emergence of aging-associated phenotypes.
}
%
% Turkish Abstract
%
\oz{
    Yaşlanma, stokastik genetik ve epigenetik değişikliklerin birikimi ile ilişkili karmaşık bir süreçtir.
    Yaşa bağlı gen anlatımı değişikliklerini inceleyen önceki bazı çalışmalar, bireyler arası heterojenliğin artmasına yönelik bir eğilimin olduğunu gösterse de,
    bu artışın tam olarak ne zaman başladığı ve fonksiyonel sonuçlarının ne olduğu sistematik olarak çalışılmamıştır.
    Bu çalışmada, 298 bireyin 17 beyin bölgesini içeren 19 mikrodizi yaş serisi veri setinin meta-analizi ile bu soruları ele aldım.
    Yaşa bağlı gen anlatımı heterojenlik değişikliklerine inceleyerek,
    yaşlanma sırasında (20 ila 98 yaş aralığı) doğum sonrası gelişim dönemine (0-20 yaş aralığı) kıyasla 
    heterojenlik tutarlılığının artışı yönünde istatistiksel olarak anlamlı bir kayma olduğu bulunmuştur.
    Bu sonuç artan heterojenliğin sadece zamanın genel etkisinden kaynaklanmadığını,
    daha ziyade yaşlanma döneminin spesifik bir etkisi olduğuna işaret ediyor.
    Ayrıca, bu tutarlı artış gösteren genlerin nöronal fonksiyon (akson rehberliği, postsinaptik uzmanlaşma) ve 
    yaşlanmayla (otofaji, mTOR sinyali) ilgili biyolojik süreçler ve yolaklar ile ilişkili olduğu bulunmuştur.
    Transkripsiyonel regülatörler (miRNA'lar ve transkripsiyon faktörleri) için gen seti zenginleştirme analizi ayrıca şunu ortaya koydu:
    artan heterojenlik sadece belirli regülatörlerle ilişkili değil, aynı zamanda regülatörlerin sayısı ile heterojenlik artışı arasında da pozitif bir ilişki var.
    
    Genel olarak, burada sunulan sonuçlar, artan bireyler arası gen anlatımı heterojenliğinin yaşlanan insan beyninin genel bir özelliği olduğunu göstermektedir.
    Ayrıca, bu artışın yaşlanma ve hastalıklarla ilgili yolaklar ve biyolojik işlevlerle ilişkili olması,
    artan heterojenliğin yaşlanma ile ilişkili fenotiplerin ortaya çıkmasına katkıda bulunabileceğini düşündürmektedir.
} 
%
% Dedication 
\dedication{\textit{To Melfi}}
%
%
% Acknowledgements   
\acknowledgments{
 Sed eget fringilla purus. Pellentesque habitant morbi tristique senectus et netus et malesuada fames ac turpis egestas. Duis id metus risus. Vestibulum tempor leo orci, non semper quam aliquet non. Nam vitae convallis dolor, lacinia mollis nunc. Vestibulum imperdiet ornare mauris eget blandit. Etiam venenatis mauris dictum urna convallis fringilla vitae non dui. Etiam sit amet venenatis turpis. Nulla facilisi. Etiam sed nulla pulvinar, ullamcorper mauris a, vulputate risus. Integer ligula nisi, porta sed dapibus ac, mattis at libero. Curabitur sodales lacus ac ante facilisis, ac condimentum ante suscipit. Proin aliquet mattis diam tempus fermentum. Phasellus quis congue ante, sed fringilla dui.

 The research I conducted as part of this thesis was previously published in~\cite{Isildak2020}. 
}

%
% End of Personal and Introductory Information
%%%%%%%%%%%%%%%%%%%%%%%%%%%%%%%%%5
\begin{document}
% Preliminaries
\begin{preliminaries}
% If you are willing to use any custom stuff before Chapters, put it here
% Such as List of Abbreviations
% Check the abbreviations.tex for a template list of abbreviations

\begin{theglossary}{LONGESTABBRV}

\item[A1C] Primary Auditory Cortex
\item[AMY] Amygdala
\item[CBC] Cerebellum
\item[DFC] Dorsolateral Prefrontal Cortex
\item[HIP] Hippocampus
\item[IPC] Posterior Inferior Parletal Cortex
\item[ITC] Inferior Temporal Cortex
\item[M1C] Primary Motor Cortex
\item[MD] Mediodorsal Nucleusofthe Thalamus
\item[MFC] Medial Prefrontal Cortex
\item[OFC] Orbital Prefrontal Cortex
\item[PFC] Prefrontal Cortex
\item[S1C] Primary Somatosensory Cortex
\item[STC] Superior Temporal Cortex
\item[STR] Striatum
\item[V1C] Primary Visual Cortex
\item[VFC] Ventrolateral Prefrontal Cortex
\item[RMA] Robust Multi-Array Analysis 
\item[PCA] Principal Component Analysis
\item[NES] Normalized Enrichment Scores
\item[GO] Gene Ontology
\item[KEGG] Kyoto Encyclopedia of Genes and Genomes

\end{theglossary}

% End of Preliminaries
\end{preliminaries}
%   
% Latex content Goes Here 
% 
%

\setlength{\parindent}{0em}
\setlength{\parskip}{10pt}

% You can add as many chapters
% CHAPTER 1
\chapter{Introduction}
\label{chp:b1}

\section{Aging}
Aging can be defined as a time-dependent deterioration of multiple biological functions and processes.
Decreasing the capacity of an organism to maintain homeostasis, 
aging is associated with an increased vulnerability to many diseases including cancer, cardiovascular and neurodegenerative disorders~\cite{Niccoli2012}.

Aging is accompanied by the accumulation of damage at all levels of an organism, from simple molecules to organs.
It is well known that replication, transcription and translation are error prone processes.
Furthermore, the protective functions against these damages are also known to be vulnerable and tend to produce other forms of damages~\cite{Gladyshev2016}.
These damages are exacerbated during the aging period and further contribute to aging and age-associated diseases.
While both stochastic and deterministic factors contribute to this damage, 
the deterministic (i.e., genetic) component is considered to be more important for distantly related species.
The effect of the genetic component, for example, can be observed in the comparison of lifespans of a human and a mouse living in the same environment.
The stochastic component, on the other hand, are suggested to be an important factor in explaning the differences within homogenous populations~\cite{Gladyshev2016}.
The stochastic factors account for the damage to nucleic acids (i.e., DNA), proteins, lipids and metaboites, as well as other age-related deleterious changes.

\subsection{Start of aging period}
Although, in everyday life, the age of 65 years old is often used to denote the beginning of the old age, the reason of choosing this age is actually historical, rather than being biological~\cite{Covey1992}.
In the biological context, however, the age-related changes were considered to manifest themselves after the organism reaches its maximum reproductive capacity~\cite{Vijg2009}.
In human populations, the age of 20 years is approximately correspond to the age of reproductive maturity~\cite{Walker2006}.

Moreover, earlier studies were also showed that the aging-associated structural changes in human brain begin to exhibit themselves when the individuals in their 20s.
These changes include declines in regional brain volumes~\cite{Sowell2003}, myelin integrity~\cite{Sullivan2006}, cortical thickness~\cite{Salat2004, Magnotta1999},
serotonin receptor binding~\cite{Sheline2002}, and concentration of several brain metabolitees~\cite{Salthouse2009, Kadota2001}.
Recent studies analyzing transcriptional patterns in human brain further revealed that the age of 20 years is a turning point in age-related gene expression trajectories,
suggesting that it roughly corresponds to point at which the aging-associated changes started.

\subsection{Hallmarks of aging}
In their 2013 paper, López \textit{et al.} reviewed 9 hallmarks of aging that were suggested to be the contributors of aging~\cite{Lopez2013}.
The authrors used three criteria while considering candidate hallmarks. 
First, a hallmark must exhibit itself during normal aging.
Second, experimental worsening of a hallmark must accelerate aging.
Third, experimental enhancement of a hallmark must slow down the aging process. 

Moreover, the hallmarks were further divided into 3 categories:
\begin{enumerate}
    \item \textbf{Primary hallmarks}, including genomic instability, telomere attrition, epigenetic alterations and loss of proteostasis,
    were considered to be primary causes of damage.
    \item \textbf{Antagonistic hallmarks}, including deregulated nutrient sensing, mitochondrial dysfunction and cellular senescence,
    were promoted (or accelerated) by the primary hallmarks, and they contribute to further accumulation of damages.
    \item \textbf{Integrative hallmarks}, including stem cell exhaustion and altered intercellular communication, 
    were suggested to arise due to accumulation of damages caused by the primary and antagonistic hallmarks,
    and they have effect on tissue homeostasis and function.
\end{enumerate}

\begin{figure}[h]
    \centering
    \includegraphics[width=.9\textwidth]{figures/figure1_1.jpeg}
    \caption{The hallmarks of aging~\cite{Lopez2013}.
    }~\label{fig:fig1.1}
\end{figure}

Each hallmark of aging is explained in detail in the following subsections.
\subsubsection{Genomic instability}~\label{hmark:genomic.instability}
Genomic instability was suggested to be one of the major stochastic mechanism of aging. 
Many studies previously demonstrated the accumulation of somatic mutations during aging in human and other model organisms~\cite{Moskalev2013,Lodato2018,Lombard2005,Vijg2004,Lu2004}.
The cumulative effect of somatic mutations further disturb the normal functioning of essential genes and transcriptinal pathways, contributing to aging and aging-associated disseases.
Moreover, deficincies in the DNA repair mechanism were found to be associated with accelarated aging, whereas experimental reinforcement of repair mechanism resulted in delayed aging.

\subsubsection{Telomere attrition}
Telomeres are repetitive and protective DNA sequences found at the end of the chromosomes.
Since most mammalian cells do not express telomerase, an enzyme responsible for maintaining telomere length, 
telomeres tend to shorten as the cell divide (i.e., as the organism age). 
Therefore, telomere length limits the proliferation capacity of somatic cells that do not express telomerase enzyme~\cite{Blasco2007}.
Moreover, telomere dysfunction was shown to be associated with accelarated aging~\cite{Armanios2009}, 
whereas experimental induction of telomerase suggested to lead to delayed aging~\cite{Tomas2008}.

\subsubsection{Epigenetic alterations}~\label{hmark:epigenetic}
A number of epigenetic alterations including alterations in DNA methlation patterns, chromatin remdeling and post-transcriptional modifications of histones, 
were suggested to constitute aging-associated epigenetic marks, significantly affecting the normal functioning of cells.
The most notable affect of age-related epigenetic alterations is on transcriptinal outcomes, given the key role of epigenetic factors in transcriptinal regulation.
It was suggested that epigenetic alterations may cause abnormal production and maturation of some mRNAs, and even further leading to increased transcriptional variation~\cite{Lopez2013}.

\subsubsection{Loss of proteostasis}~\label{hmark:proteosis}
Proteostasis, mechanisms that involve in preserving the stabilitty and functionality of proteome, is suggested to be altered during aging~\cite{Koga2011}. 
Age-associated impairment of proteostasis leads to continouos expression of misfolded and aggregated proteins, 
whose accumulation during aging further contributes to the development of age-associated diseases, including Alzheirmer's disease and Parkinson's disease~\cite{Powers2009}.

\subsubsection{Deregulated nutrient sensing}
Nutrient-sensing pathways play important role in sensing the presence or absence of extra- annd intra-cellular nutrients, and they further regulates their intake.
The Insulin/Insuline-like growth factor signaling (IIS) pathway is one of the nutrient-sensing pathways that was found to be regulating aging.
The downstream intracellular effectors of the IIS pathway include AKT, mTOR and FoxO, all of which were suggested to be associated with aging~\cite{Fontana2010, Barzilai2012, Kenyon2010}.
A decreased activity of the IIS and mTOR signaling pathways, for example, was found to extend life span in many model organisms~\cite{Fontana2010}. 

\subsubsection{Mitochondrial dysfunction}~\label{hmark:mt.dysfunction}
During the normal aging, the mitochondrial machinery becomes rusty, leading to increased electron leakage and reduced ATP generation~\cite{Green2011}.
In addition to accumulation of damages in the nuclear DNA (see \textbf{Section~\ref{hmark:genomic.instability}}), 
mitochondrial DNA (mtDNA) is also considered to be vulnerable to the somatic mutations due to limited repair mechanisms and oxidative microenvironment,
leading to impaired functionality of mitochondria.
Moreover, age-related increase in reactive oxygen species (ROS) was suggested to cause global cellular damage after a certain treshold, 
further contributing to the emergence of aging-associated phenotypes~\cite{Hekimi2011}.

\subsubsection{Cellular senescence}~\label{hmark:cell.senes}
Cellular senescence can be defined as permanent arrest of the cell cycle.
It can be triggered by a number of factors including, telomore shortening, DNA damage and a number of mitogenic alterations.
Number of senescent cells were shown to increase with age.
Although cellular senescence is originally a protective mechanism preventing the proliferation of damaged cells,
their accumulation during aging results in deleterious effects on tissue homeostasis, further contributing to aging~\cite{Lopez2013}.

\subsubsection{Stem cell exhaustion}
It is long known that aging is accompanied by a decline in stem cell numbers and renewal capacity, 
contributing to declined homeostatic and regenerative capacity of aged tissues~\cite{Oh2014}.
The factors contributing to stem cell exhaustion includes
DNA damage (\textbf{Section~\ref{hmark:genomic.instability}}), 
epigenetic alterations (\textbf{Section~\ref{hmark:epigenetic}}), 
aggregaion of damaged proteins (\textbf{Section~\ref{hmark:proteosis}}),
accumulation of toxic metabolites (i.e., ROS) and
mithochondrial dysfunction (\textbf{Section~\ref{hmark:mt.dysfunction}}).
Moreover, experimental rejuvenation of stem cells was found to reset the aging clock, 
meaning that it has potential to reverse aging-associated phenotypes~\cite{Rando2012}.

\subsubsection{Altered intercellular communication}
In addition to cell-autonomous hallmarks, the last hallmark of aging is related to altered communication of cells 
in terms of endocrine, neuroendocrine and neuronal signaling~\cite{Russell2007}.
A harmonious intercellular communication was suggested to be a key factor for stress response, cell survival and maintaining homeostasis~\cite{Tan2021}.
The aging period is associated with increased inflammatory reactions, decreased immunosyrveillance and changed extracelular environment,
all of which contribute to the deregulation of neurohormonal signaling.
Specifically, senescent cells were known to show an altered secretome, which is rich in proinflammatory cytokines, 
which in turn contributes to the emmergence of aging-associated phenotypes~\cite{Childs2016, Kuilman2010}.

\section{Age-related gene expression changes}
As high-throughput technologies become more affordable and widely accessible, 
the past two decades have seen a dramatic increase in studies that focus on gene expression changes in brain during the aging period.
One of the earlier studies conducted by Lu and colleagues found that the expression levels of the genes 
that play important role in synaptic plasticity and mitochondrial were tend to decrease in aging~\cite{Lu2004}. 
They further demonstrated that this decrease is also accompanied by increased promotor damage, 
suggesting that DNA damage may reduce the expression of genes involved in neuronal functioning, 
possibly contributing to the emergence of aging-associated pathologies.

One common gene expression signature of aging is the downregulation of genes encoding mithochondrial ribosomal proteins and components of the electron transport chain.
In their 2013 study, Kumar \textit{et al.} analyzed microarray data generated from the frontal lobe of the cerebral cortex and cerebellum,
and found that genes encoding mitochondrial components tend to be downregulated during aging~\cite{Kumar2013}.
Similar trends were also observed in other model organisms including rodents, flies and worms, and across different tissues from skin to muscle~\cite{Frenk2018},
suggesting that the downregulation of genes encoding mithochondrial proteins may be a characteristic of aging, 
and may contribute to the age-related mitochondrial dysfunction (\textbf{Section~\ref{hmark:mt.dysfunction}}).




\subsection{Age-related heterogeneity changes}
It is known that the aging period is also associated with dysregulation of gene expression and mRNA processing.
There have been 


\section{Research Objectives}
Research objectives




% CHAPTER 2
\chapter{Material and Method}~\label{chp:b2}
\section{Datasets}
In this study, I analyzed 19 microarray gene expression datasets to investigate age-related gene expression heterogeneity change during development and aging.
The datasets were retrieved from 3 independent published studies, containing microarray data for the human brain~\cite{Colantuoni2011, Kang2011, Somel2011, Somel2010}.
Overall, the datasets include 1010 samples from 298 individuals spanning 17 different brain region, which are not mutually exclusive.
All datasets have samples covering whole lifespan with ages ranging from 0 to 98 years (\textbf{Figure~\ref{fig:fig2.1}}).
A summary of datasets used in this study is shown in \textbf{Table~\ref{table:table1}}.

It should also be noted that Kang2011 datasets contain samples from left and right hemispheres of the same individual.
These samples were analyzed as biological replicates, meaning that samples were not seperated into two different datasets, for three reasons.
First, it was previously suggested that left and right hemispheres of the brain may show asymmetric age-related changes~\cite{Sun2005, Dolcos2002}.
Second, the other datasets do not contain hemisphere information.
Last, previous studies analyzing this dataset, including the original study, also treated them as biological replicates~\cite{Kang2011,Donertas2017}.

Additionally, Somel2011\char`_PFC dataset included two pairs of technical replicates, between which the correlation was high. 
Therefore, the mean of expression values was used in the downstream analysis.


The datasets were downloaded from NCBI Gene Expression Omnibus (GEO) database using the accession codes given in the
\textbf{Table~\ref{table:table1}}. 
All the analysis was performed in R programming environment.

\begin{figure}[h]
\centering
\includegraphics[width=.9\textwidth]{figures/figure2_1.png}
\caption{The distribution of ages of the samples. (a) Number of samples included in age intervals. (b) Distribution of ages. The color coding reflect different data sources. }\label{fig:fig2.1}
\end{figure}

\begin{table}[ht]
\centering
\caption{The list of microarray human brain gene expression datasets.}\label{table:table1}
%% \resizebox{\textwidth}{!}{\begin{tabular}{||c c c c||} 
\begin{tabular}{|c c c c|}
 \hline
 \textbf{GEO Acc.} & \textbf{Source} & \textbf{Brain Region} & \textbf{Sample Size} \\ [0.5ex] 
 \hline\hline
 GSE30272 & Colantuoni2011 & PFC & 231 \\ 
 \hline
 GSE25219 & Kang2011 & A1C & 47 \\
 \hline
 GSE25219 & Kang2011 & AMY & 43 \\
 \hline
 GSE25219 & Kang2011 & CBC & 47 \\
 \hline
 GSE25219 & Kang2011 & DFC & 48 \\
 \hline
 GSE25219 & Kang2011 & HIP & 39 \\
 \hline
 GSE25219 & Kang2011 & IPC & 49 \\
 \hline
 GSE25219 & Kang2011 & ITC & 49 \\
 \hline
 GSE25219 & Kang2011 & M1C & 45 \\
 \hline
 GSE25219 & Kang2011 & MD & 43 \\
 \hline
 GSE25219 & Kang2011 & MFC & 50 \\
 \hline
 GSE25219 & Kang2011 & OFC & 48 \\
 \hline
 GSE25219 & Kang2011 & S1C & 46 \\
 \hline
 GSE25219 & Kang2011 & STC & 48 \\
 \hline
 GSE25219 & Kang2011 & STR & 41 \\
 \hline
 GSE25219 & Kang2011 & V1C & 48 \\
 \hline
 GSE25219 & Kang2011 & VFC & 47 \\
 \hline
 GSE22569 & Somel2011 & PFC & 23 \\
 \hline
 GSE18069 & Somel2011 & CBC & 22 \\
\hline
\end{tabular}
\end{table}

\subsection{Dataset selection}
The age-series datasets analyzed in this study are all microarray-based. 
Although there was one other RNA-Sequencing based dataset that covers whole lifespan~\cite{Mazin2013}, I chose not to include it in this anaysis for two reasons.
First, the samples were already included in the Somel2011 dataset. 
Second, it is an underpowered dataset with data from only 35 individuals that cannot reliably lead to conclusion.

There were also RNA-Sequencing datasets containing samples from only development or aging periods. 
Since combining independent development and aging datasets may confound biological effects that I aimed to examine, 
this study was limited to meta-analysis of 19 microarray-based datasets. 

\subsection{Seperating development and aging datasets}
The aim of this study to investigate age-related gene expression change during development and aging. 
Thus, all the datasets were seperated into two datasets: development (0 to 20 years of age) and aging (20 to 98 years of age).
The age of 20 years was used to seperate development and aging for the following reasons:
\begin{enumerate}
    \item The age of 20 was shown to correspond approximately to the age of reproduction in human societies~\cite{Walker2006}.
    \item Previous studies investigating age-related gene expression trajectories demonstrated that 20 years of age corresponds to a turning point of gene expression patterns\cite{Colantuoni2011, Donertas2017,Somel2010}.
    \item Earlier research connected the structural changes occuring in the human brain after the age of 20 to age-related phenotypes~\cite{Sowell2004}.
\end{enumerate}

As a result, I obtained: 
(i) one development and one aging dataset for the Colantuoni2011; 
(ii) 16 development and 16 aging datasets for the Kang2011; and
(iii) two development and two aging datasets for the Somel2011.
Overall, both development and aging datasets resulted in a comparable number of samples ($n_{development} = 441$; $n_{aging}=569$).

Moreover, it is important to note that I excluded samples from prenatal development period, 
since gene expression trajectories were shown to be discontinuous between prenatal and postnatal development period~\cite{Colantuoni2011,Kang2011}, 
and since the scope of this study is limited to investigate changes in gene expression heterogeneity during aging compared to pre-adulthood.

\section{Dataset Preprocessing}
Microarray technology is a widely used tool to quantify expression level of gene transcripts from a given sample. 
A microarray chip contains known sequences of oligonucleotides -known as probes- that are located on a solid surface.
Typically, each transcript is represented by a set of 11-20 pairs of probe, called as the probe-set of that transcript, in Affymetrix microarray platforms.
The mRNA molecules from the sample are hybridised to target probes labelled by detectable fluorochrome molecules, 
where the amount of hybridization is reflected by the light intensity levels.
The quantification of expression is then performed by measuring light intensity levels of each probe, which are stored in CEL files.

The Kang2011 and Somel2011 datasets were generated by Affymetrix HuEx-1\char`_0-st and HuGene-1\char`_0-st microarray platforms, respectively. 
Colantuoni2011 dataset, on the other hand, was generated using HEEBO-7 set (Human 49K oligo array), which is an Illumina based array. 
Since there is no public R library available to process Illumina based data from Colantuoni2011, 
I used the expression data preprocessed by the authors of the original study~\cite{Colantuoni2011}. 
For the datasets from Kang2011 and Somel2011 sources, I downloaded CEL files from GEO database~\cite{Barrett2013}. 
The preprocessing of Kang2011 and Somel2011 datasets can be summarized in four steps: (1) RMA convolution, (2) probe-set summarization,
(3) log2 transformation, and (4) quantile normalization. For the Colantuoni2011 dataset, quantile normalization was performed on the preprocessed data.


\subsection{RMA correction}
The very first step of a microarray analysis is the removal of noise and biases from the raw data obtained from light intensities.
There can be a number of factors contributing background errors, such as optical noise, unspecific hybridization and incomplete washing~\cite{Bengtsson2006}. 
Nevertheless, low-level preprocessing and normalization, having a significant effect on the downstream analysis, 
were suggested to be one of the most important step in any microarray data analysis. 

In this study, background normalization was performed by the Robust Multiarray Average (RMA) convolution method, 
which is a one of the most widely used method to perform background normalization on microarray data. 
The RMA method involves the removal of technical artefacts so that the measurements from neighbouring probes do not interfere with each other~\cite{Irizarry2003}.

Apart from background normalization, the RMA algorithm also performs probe to probe-set summarization. 
Since each transcript is represented by a set of 11-20 probes, it is necessary to summarize probe-level data into probe-sets, 
by grouping probes corresponding the same transcript. I used the R ``oligo'' library to perform RMA correction~\cite{Carvalho2010}.
As previously stated, RMA correction was performed only on Kang2011 and Somel2011 datasets.

\subsection{Probe-set summarization}

\subsection{Log2 transformation}

\subsection{Quantile normalization}

\begin{figure}[h]
\centering
\includegraphics[width=.9\textwidth]{figures/figure2_2.png}
\caption{Summary of preprocessing steps. Each histogram shows Somel2011\char`_CBC dataset at different steps.}~\label{fig:fig2.2}
\end{figure}

\subsection{Batch-effect correction}

\subsection{Scaling}

\section{Age-related expression change}

\begin{equation}
    Expr_i = \beta_{i0} + \beta_{i1} * Age^{1/4} + \epsilon_i
    \label{eq:exp_change}
\end{equation}

\section{Age-related heterogeneity change}


\begin{figure}[h]
\centering
\includegraphics[width=.9\textwidth]{figures/figure2_3.png}
\caption{Summary of the method used to calculate age-related expression change (left panels) and age-related expression heterogeneity change (right panels) during development (a) and aging (b). Adapted from (Isildak et al., 2020)}\label{fig:fig2.3}
\end{figure}

\section{Permutation test}

\subsection{Expected heterogeneity consistency}

\section{Functional analysis}
asfgafs

\section{Effect of sex-specific gene expression}



\begin{equation}
{\bf y}_k = {\bf x}_k * {\bf P}_k
\end{equation}


























% CHAPTER 3
\chapter{Results}
\label{chp:b3}

In this study, I investigated the age-related gene expression heterogeneity change by analyzing 19 microarray datasets
containing 1,010 samples from diverse brain regions of 298 individuals, and covering whole lifespan.
To compare heterogeneity change occuring during before and after adulthood, each dataset were first seperated into two datasets: development dataset 
(including samples with ages ranging from 0 to 20 years old) and aging dataset (including samples with ages ranging from 20 to 98 years old).

\section{Age-related gene expression change}
While the aim of this study to investigate age-related changes in gene expression heterogeneity, I first sought to characterize age-related change in gene expression.
Linear regression was performed to characterize the relationship between gene expression and age for each gene and dataset separately (\textbf{Section \ref{sub:exp-change}} ).

\begin{figure}[h]
\centering
\includegraphics[width=.9\textwidth]{figures/figure3_1.png}
\caption{Age-related gene expression changes (Adapted from Isildak et al., 2020)}
\label{fig:fig3.1}
\end{figure}

\begin{figure}[h]
    \centering
    \includegraphics[width=.9\textwidth]{figures/figure3_2.png}
    \caption{Age-related gene expression changes}
    \label{fig:fig3.2}
    \end{figure}

\section{Age-related expression heterogeneity change}




% CHAPTER 4
\chapter{Discussion}
\label{chp:b4}

In this study, I aimed to investivage the changes in gene expression heterogeneity during development and aging periods.
The dataset that I analyzed included 19 microarray dataset containing gene expression measurements for human brain from 3 independent sources.
Overall, 1,010 samples from 17 different brain regions of 298 individuals (\textbf{Table~\ref{table:table1}}, \textbf{Figure~\ref{fig:fig2.1}}).

The datasets, containing samples covering whole lifespan (ages from 0 to 98 years old), 
were first divided into development and aging datasets,
using the age of 20 years as a seperation point (see \textbf{Section~\ref{subsec:dset.seperation}}), 
which was previously shown to be the global turning point of gene expression trajectories~\cite{Donertas2017, Colantuoni2011, Somel2010}.
Overall, I obtained 19 development datasets including samples whose ages range from 0 to 20 years old (n = 441).
It is also important to note that pre-natal samples were excluded from the downstream analysis
since the gene expression trajectories are suggested to be discontinous between pre- and post-natal development, 
and the scope of this study was to compare heterogeneity changes during postnatal development and aging.
The aging datasets (n = 19 datasets), on the other hand, included samples whose ages range from 20 to 98 (n = 569).
Only the common genes (i.e., the genes for which I have measurement across all datasets) were included in the downstream analysis (n = 11,137).

Using the advantage of having multiple datasets, this study focused on consistent changes that are shared across the datasets, 
rather than focusing on significant changes in a single dataset, which is highly dependent on the sample sizes. 
Thus, this approach was able to capture weak but shared signals that would otherwise fail to pass the significance treshold in individual datasets.

\section{Correlations among datasets in expression and heterogeneity changes}
After performing preprocessing on microarray datasets (see \textbf{Section~\ref{sec:dset.preprocess}}), 
I first sought to characterize the age-related changes in gene expression, 
by performing a linear regression analysis between scaled expression values and 
fourth root of ages in days as shown in the left panels of \textbf{Figure~\ref{fig:fig2.3}}.
The $\beta_{i1}$ values obtained from \textbf{Equation~\ref{eq:exp_change}} were considered as the measure of age-related expression change.
The regression analysis was performed for each gene and for each time period, seperately (see \textbf{Section~\ref{sec:exp-change}} for details).

Then I investigated the coordination in expression change between all possible pairs of datasets by calculating Spearman's correlation coefficient 
and found that the correlation among development datasets (median correlation coefficient = 0.56) 
is significantly higher than the correlation among aging datatsets (median correlation coefficient = 0.43, permutation test p-value = 0.003).
Furthermore, more genes showed significant changes during development compared to aging period, and they mostly tended to decrease in expression (\textbf{Figure~\ref{fig:fig3.1}}).
One possible explanation of these results might be related to stochastic nature of aging. 
As previously suggested, the accumulation of random detrimental effects (i.e., mutations) during aging may cause reduced gene expressions, 
and in turn causing an increased level of heterogenety in aging~\cite{Lu2004}.
Consistent with earlier findings, my initial analysis of gene expression changes also suggests that the changes in development are well-regulated,
and further supports the view of aging as a stochastic process.

Next, the change in gene expression heterogeneity was characterized 
by performing Spearman's correlation test between absolute value of residuals obtained from \textbf{Equation~\ref{eq:exp_change}} 
and the fourth root of age for each gene and time period seperately (see \textbf{Section~\ref{sec:het-change}}).
Analyzing the correlations among development and aging datasets in heterogeneity change,
I found that aging datasets display a higher correlation compared to development datasets, 
reflecting a more consistent heterogeneity change in aging.
A further analysis of genes showing significant heterogeneity changes revealed that 
there are more genes showing significant changes in heterogeneity during aging, compared to development (\textbf{Figure~\ref{fig:fig3.2}b}). 
Moreover, the significant changes in heterogeneity are mostly in the positive direction during aging, suggesting an increase in heterogeneity.

\section{Increased heterogeneity consistency in aging}
Having observed more consistent heterogeneity change and more significant heterogeneity increase in aging, 
I next investigated heterogeneity changes in individual datasets and found an overall increase in heterogeneity during aging 
(i.e., 18 of 19 aging datasets display higher median heterogeneity change compared to development, see \textbf{Figure~\ref{fig:fig3.3}a}). 
An analysis of consistent heterogeneity change further revealed that there is a significant shift towards increased heterogeneity consistency during aging compared to expectation,
while no such shift was observed for development datasets (\textbf{Figure~\ref{fig:fig3.3}c}).

There are number of factors that can explain increased heterogeneity during aging compared to development.
First, a number of studies previously demonstrated that the stochastic accumulation of somatic mutations may cause genomic instability,
which may in turn lead to increased heterogeneity in aging period~\cite{Lu2004, Vijg2004, Lodato2018, Lombard2005}.
Second, Cheung \textit{et al.}, analyzing a twin cohort, demonstrated the stochastic nature of age-related changes in chromatin,
leading to increase variation between both individuals and cells in aging~\cite{Cheung2018}.
The third factor might be the transcriptional regulation, which was suggested to be stochastic due to randomness of biochemical reactions~\cite{Maheshri2007, Barroso2018}.
Previous studies found that the variability in gene expression is positively correlated with the number of transcription factors that control is regulation~\cite{Barroso2018, Sharon2014}.
While the first two factors could not be tested in this study since the datasets did not contain somatic mutation or epigenetic regulation information, 
I also found a mainly positive correlation between number of regulators and heterogeneity change during aging (\textbf{Figure~\ref{fig:fig3.7}}).
Therefore, the results obtained in this study also supported the view that increased heterogeneity in aging may be associated with the transcriptional regulation.

\section{Increased heterogeneity may have important functional consequences}
Then, I investigated the functional consequences of increased heterogeneity in aging by conducting gene set enrichment analysis for GO biological processes and KEGG pathways.
The significantly enriched KEGG pathways included many pathways that were known to be important in aging,
including longevity regulating pathway, authophagy, and mTOR signalin pathways (\textbf{Figure~\ref{fig:fig3.3}}).
Moreover, the significantly enriched GO terms also included terms that are related to aging and aging-related diseases,
suggesting functional significance of increased heterogeneity in aging and aging-related diseases.
Additionally, GO terms related to neuronal and synaptic functions were also enriched for genes showing increased heterogeneity consistency during aging,
indicating the potential role of increased heterogeneity in age-related cognitive decline, which was suggested to be mainly a result of synaptic dysfunction~\cite{Morrison2012}.

I also performed gene set enrichment analysis for transcriptional regulators. Transcription factors found to be significantly associated with increased heterogeneity during aging included FoxO and EGR family of transcription factors, 
which were shown to be regulating genes important for synaptic homeostasis, stress resistance, cell cycle arrest and apoptosis.
Combined, gene set enrichment analysis revealed a potentially important role of increased heterogeneity in human brain aging.

\section{Increased heterogeneity is a biological signal}
I next confirmed that the observed increase in heterogeneity consistency is a biological signal, rather than being a technical artifacts or a result of low statistical power, 
given the similar sample sizes of development and aging periods (\textbf{Figure~\ref{fig:fig2.1}}).

One technical factor that can explain increased heterogeneity during aging might be related to dependence between mean and variance, 
where accompanying increase in expression levels may cause increased variance, which in turn detected as increased heterogeneity.
To address this issue, I performed correlation test between heterogeneity changes and expression changes for each dataset and each period, 
and found that the correlation coefficients calculated for aging datasets are mostly negative (\textbf{Figure~\ref{fig:fig3.3}b}), 
suggesting that observed increase in heterogeneity was not caused by mean-variance dependence.

Another technical factor that can cause increased inter-individual variance may be related to post-mortem interval (PMI), 
which measures the time between death and sample collection.
It was previously suggested that PMI-related mRNA degradation is gene-specific, leading to a bias in downstream analysis~\cite{Zhu2017}.
To confirm that increased heterogeneity was not a result of PMI-related mRNA degradation,
I used previously identified 107 PMI-associated genes~\cite{Zhu2017}, 75 of which were included in this analysis.
Specifically, I tested if the 75 PMI-associated genes show more increase in heterogeneity during aging,
and found that only 2 of 147 consistent genes were PMI-associated, 
suggesting that PMI by itself was not enough to explain observed increase in heterogeneity (\textbf{Figure~\ref{fig:a6.1}}).

One other factor that can affect the main results presented here might be related to age scales.
As previously stated, the fourth root of age scale was used in this study to obtain relatively uniform distribution of ages across the lifespan.
However, whether the observed increase in heterogeneity depends on the use of specific age scales remained unanswered.
To assess the effect of using different age scales on the downstream analysis, I repeated the analysis using 3 additional age scales: 
(1) age in days, (2) age in log2 scale, and (3) age in years (\textbf{Figure~\ref{fig:a6.2}}).
Overall, I found that using different age scales also yield a quantitatively similar results.
In fact, the use of log2 age scale resulted in a higher number of genes showing consistent increase in heterogeneity across all 19 datasets (\textbf{Figure~\ref{fig:a6.2}b}, lower left panel).
Nevertheless, this analysis indicated that the observed increase in heterogeneity is not a result of the use of specific age scale.

It was previously suggested a sex-specific difference in human brain aging, where males showed more changes in gene expression~\cite{Berchtold2008}.
In this analysis, however, both males and females were combined to calculate expression and heterogeneity changes, raising a question about possible confound of sex with age.
To address this question, I retrieved the real values of residuals obtained from \textbf{Equation~\ref{eq:exp_change}} 
(not absolute values) for 147 genes showing consistent increase in heterogeneity.
Then, for each gene and each aging dataset, two sample Wilcoxon test was performed on residuals to test if there is a significant difference between males and females.
The obtained p-values were corrected for multiple testing by B{\&}H method (\textbf{Section~\ref{subsec:p.adjust}}).
Overall, I found that there are only 15 out of 147 consistent genes show significant difference between sexes in at least one dataset (\textbf{Figure~\ref{fig:a6.3}}),
suggesting that the increased heterogeneity cannot be explained solely by sex-specific differences in brain aging.

In this study, a permutation scheme that takes into account the dependency of Kang2011 and Somel2011 
datasets was employed to calculate expectation of heterogeneity consistency (\textbf{Section~\ref{sec:perm}}).
By also using random permutations to calculate expected consistency in heterogeneity increase, 
I found that the scheme used in this study was more strict than the random permutations (\textbf{Figure~\ref{fig:a6.4}}).

One important assumption of this study is that the relationship between scaled expression levels and fourth root of age is linear,
since linear regression was used to characterize the age-related expression change.
To ensure that this assumption did not have significant effect on the downstream results, 
I re-calculated heterogeneity changes using the residuals obtained from loess regression and 
found a high correlation between heterogeneity changes calculated using linear regression and loess regression (\textbf{Figure~\ref{fig:a6.5}}).
Yet, the heterogeneity changes calculated from loess regression did not included in the downstream analysis
since both the model parameters and sample sizes have significant effect on the estimates of loess regression.

The last factor that can cause the observed increase in heterogeneity might be related to the outliers in the datasets.
For example, one older individual having too low or high expression value (i.e., having higher absolute value of residual) 
can drive the heterogeneity estimates up.
To investigate the effect of outliers, I plotted the absolute value of residuals for 147 consistent genes (\textbf{Figure~\ref{fig:a6.5}}).
A visual inspection revealed that there was no significant outlier sample that can explain the observed increase in heterogeneity.

Overall, these extra analysis demonstrated that the increased heterogeneity reported in this study cannot be explained by low statistical power and technical factors,
but rather it is indeed a biological signal.

\section{Limitations \& future perspectives}
\begin{enumerate}
    \item Microarray platform, unlike RNA-Sequencing, is unable to measure absolute abundance of mRNAs. 
    Rather, the light intensities obtained from microarray platform reflect the relative expression levels.
    In their 2018 paper, Davie \textit{et al.} found that the total abundance of mRNAs tends to decrease with age~\cite{Davie2018}.
    Moreover, it was periously shown that the genes with lower genes expression values are susceptible to have higher variance~\cite{Aris2004}.
    In this respect, the dataset analyzed in this study was unable to detect the contribution of mRNA decay and
    further researches are needed to understand the contribution of total mRNA decay to observed increase in gene expression heterogeneity.
    \item Another limitation is also related to microarray datasets containing bulk mRNA expression data.
    Since the expression datasets analyzed in this study contains the average measurements for many cells,
    the results presented here only reflects increased heterogeneity among individuals, not cells.
    Future researches using single cell RNA-Sequencing data are required to investigate heterogeneity changes between cells.
    \item Although 19 different datasets were analyzed in this study, it is important to note that they were originated from 3 independent sources,
    where Somel2011 and Kang2011 datasets contain measurements from different brain regions of the same individual.
    \item Although a significant overall shift was observed towards increased heterogeneity consistency during aging (\textbf{Figure~\ref{fig:fig3.3}c, lower panel}),
    a gene set that become significantly more heterogeneous across all datasets was not confidently identified due to ~40\% true positive rate.
    \item One other limitation is related to unequal sample sizes across datasets. 
    Specifically, the Colantuoni2011 dataset had markedly higher sample size compared to all other datasets,
    leading to higher statistical power, and subsequently identification of more significant genes in both expression (\textbf{Figure~\ref{fig:fig3.1}c}) 
    and heterogeneity changes (\textbf{Figure~\ref{fig:fig3.2}c}).
    \item While two possible explanations of increased heterogeneity are related to accumulation of somatic mutations and epigenetic regulations,
    I couldn't test their effect due to lack of data. 
    More comprehensive studies incorporating different types of data are needed to reveal the source of increased heterogeneity.
\end{enumerate}

% CHAPTER 5
\chapter{Conclusion}
\label{chp:b5}

Aging is a complex process characterized by a gradual functional decline.
Moreover, aging is considered to pose a major risk factor for many diseases, including cancer, and cardiovascular and neurodegenerative disorders.
Transcriptome studies focusing on age-related changes in the human brain have been offering novel insights into the understanding of underlying mechanisms.

In this study, I conducted a meta-analysis of 19 microarray datasets containing 1,010 samples from 17 brain regions and covering the whole lifespan.
Specifically, I investigated the changes in inter-individual gene expression heterogeneity during aging in comparison to the development period.
The main findings of this study were summarized in the following bullet points below.

\begin{itemize}
    \item There are more genes showing significant changes in gene expression during development compared to aging.
    \item In development, the majority of genes showing significant expression change decrease in expression.
    \item Development datasets show higher coordination in gene expression changes, compared to aging datasets.
    \item Gene expression heterogeneity consistently increases with age during aging (20 to 98 years of age) but not in postnatal development (0 to 20 years of age).
    \item The heterogeneity increase observed in aging comes with physiological consequences, 
    such that the genes showing consistent effects are associated with biological processes important for life- and health-span regulation 
    (e.g. autophagy, mTOR signaling), as well as for cognitive functions (e.g. axon guidance, postsynaptic specialization).
    \item Not only specific regulators (miRNAs and transcription factors) but also the number of regulators is positively associated with consistent changes in heterogeneity. 
\end{itemize}


%
% References in Bibtex format goes into below indicated file with .bib extension
% \bibliographystyle{ieeetr} 
\bibliographystyle{abbrv}
\bibliography{biblio.bib}

% You can use full name of authors, however most likely some of the Bibtex entries you will find, will use abbreviated first names
% If you don't want to correct each of them by hand, you can use abbreviated style for all of the references

% if you have more that one appendix, then use \appendices, otherwise use 
\appendices
\input{appendices/appendix1.tex}
\input{appendices/appendix2.tex}
\input{appendices/appendix3.tex}
\input{appendices/appendix4.tex}
\input{appendices/appendix5.tex}
\input{appendices/appendix6.tex}
\end{document}
