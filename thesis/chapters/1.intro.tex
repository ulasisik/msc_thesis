% CHAPTER 1
\chapter{Introduction}
\label{chp:b1}

\section{Aging}
Aging can be defined as a time-dependent deterioration of multiple biological functions and processes.
Decreasing the capacity of an organism to maintain homeostasis, 
aging is associated with an increased vulnerability to many diseases including cancer, cardiovascular and neurodegenerative disorders~\cite{Niccoli2012}.

Aging is accompanied by the accumulation of damage at different levels of an organism, from simple molecules to organs.
It is well known that replication, transcription and translation are error-prone processes.
Furthermore, the protective functions against these damages are also known to be vulnerable and tend to produce other forms of damages~\cite{Gladyshev2016}.
These damages are exacerbated during the aging period and further contribute to aging and age-associated diseases.
While both stochastic and deterministic factors contribute to this damage, 
the deterministic (i.e., genetic) component is considered to be more important for distantly related species.
The effect of the genetic component, for example, can be observed in the comparison of the lifespans of a human and a mouse living in the same environment.
The stochastic component, on the other hand, is suggested to be an important factor in explaining the differences within homogenous populations~\cite{Gladyshev2016}.
The stochastic factors account for the damage to nucleic acids (i.e., DNA), proteins, lipids and metabolites, as well as other age-related deleterious changes.

\subsection{Start of aging period}
Although in everyday life, the age of 65 years old is often used to denote the beginning of old age, 
the reason for choosing this age is actually historical, rather than being biological~\cite{Covey1992}.
In the biological context, however, the age-related changes were considered to manifest themselves after the organism reaches its maximum reproductive capacity~\cite{Vijg2009}.
In human populations, the age of 20 years approximately corresponds to the age of reproductive maturity~\cite{Walker2006}.

Moreover, earlier studies also showed that the aging-associated structural changes in the human brain begin to exhibit themselves when individuals are in their 20s.
These changes include declines in regional brain volumes~\cite{Sowell2003}, myelin integrity~\cite{Sullivan2006}, cortical thickness~\cite{Salat2004, Magnotta1999},
serotonin receptor binding~\cite{Sheline2002}, and concentration of several brain metabolitees~\cite{Salthouse2009, Kadota2001}.
Recent studies analyzing transcriptional patterns in the human brain further revealed that the age of 20 years is a turning point in age-related gene expression trajectories,
suggesting that it roughly corresponds to the point at which the aging-associated changes started~\cite{Somel2010, Colantuoni2011, Donertas2017}.

\subsection{Hallmarks of aging}
In their 2013 paper, López \textit{et al.} reviewed 9 hallmarks of aging that were suggested to be the contributors to aging~\cite{Lopez2013}.
The authors used three criteria while considering candidate hallmarks. 
First, a hallmark must exhibit itself during normal aging.
Second, experimental worsening of a hallmark must accelerate aging.
Third, experimental enhancement of a hallmark must slow down the aging process. 

Moreover, the hallmarks were further divided into 3 categories:
\begin{enumerate}
    \item \textbf{Primary hallmarks}, including genomic instability, telomere attrition, epigenetic alterations and loss of proteostasis,
    were considered to be primary causes of damage.
    \item \textbf{Antagonistic hallmarks}, including deregulated nutrient sensing, mitochondrial dysfunction and cellular senescence,
    were promoted (or accelerated) by the primary hallmarks, and they contribute to further accumulation of damages.
    \item \textbf{Integrative hallmarks}, including stem cell exhaustion and altered intercellular communication, 
    were suggested to arise due to the accumulation of damages caused by the primary and antagonistic hallmarks,
    and they further affect tissue homeostasis and function.
\end{enumerate}

\begin{figure}[h]
    \centering
    \includegraphics[width=.9\textwidth]{figures/figure1_1.jpeg}
    \caption{The hallmarks of aging~\cite{Lopez2013}.}
    \label{fig:fig1.1}
\end{figure}

Each hallmark of aging is explained in detail in the following subsections.
\subsubsection{Genomic instability}~\label{hmark:genomic.instability}
Genomic instability was suggested to be one of the major stochastic mechanisms of aging. 
Many studies previously demonstrated the accumulation of somatic mutations during aging in humans and other model organisms~\cite{Moskalev2013, Lodato2018, Lombard2005, Vijg2004, Lu2004}.
The cumulative effect of somatic mutations further disturbs the normal functioning of essential genes and transcriptional pathways, contributing to aging and aging-associated diseases.
Moreover, deficiencies in the DNA repair mechanism were found to be associated with accelerated aging, whereas experimental reinforcement of the repair mechanism resulted in delayed aging.

\subsubsection{Telomere attrition}
Telomeres are repetitive and protective DNA sequences found at the end of the chromosomes.
Since most mammalian cells do not express telomerase, an enzyme responsible for maintaining telomere length, 
telomeres tend to shorten as the cell divide (i.e., as the organism age). 
Therefore, telomere length limits the proliferation capacity of somatic cells that do not express telomerase enzyme~\cite{Blasco2007}.
Moreover, telomere dysfunction was shown to be associated with accelerated aging~\cite{Armanios2009}, 
whereas experimental induction of telomerase was suggested to lead to delayed aging~\cite{Tomas2008}.

\subsubsection{Epigenetic alterations}~\label{hmark:epigenetic}
A number of epigenetic alterations including alterations in DNA methylation patterns, chromatin remodeling and post-transcriptional modifications of histones, 
were suggested to constitute aging-associated epigenetic marks, significantly affecting the normal functioning of cells.
The most notable effect of age-related epigenetic alterations is on transcriptional outcomes, given the key role of epigenetic factors in transcriptional regulation.
It was suggested that epigenetic alterations may cause abnormal production and maturation of some mRNAs, and even further lead to increased transcriptional variation~\cite{Lopez2013}.

\subsubsection{Loss of proteostasis}~\label{hmark:proteosis}
Proteostasis, mechanisms that involve preserving the stability and functionality of the proteome, is suggested to be altered during aging~\cite{Koga2011}. 
Age-associated impairment of proteostasis leads to continuous expression of misfolded and aggregated proteins, 
whose accumulation during aging further contributes to the development of age-associated diseases, including Alzheimer's disease and Parkinson's disease~\cite{Powers2009}.

\subsubsection{Deregulated nutrient sensing}
Nutrient-sensing pathways play important role in sensing the presence or absence of extra- and intra-cellular nutrients, and they further regulate their intake.
The Insulin/Insulin-like growth factor signaling (IIS) pathway is one of the nutrient-sensing pathways that was found to be regulating aging.
The downstream intracellular effectors of the IIS pathway include AKT, mTOR and FoxO, all of which were suggested to be associated with aging~\cite{Fontana2010, Barzilai2012, Kenyon2010}.
A decreased activity of the IIS and mTOR signaling pathways, for example, was found to extend the lifespan in many model organisms~\cite{Fontana2010}. 

\subsubsection{Mitochondrial dysfunction}~\label{hmark:mt.dysfunction}
During normal aging, the mitochondrial machinery becomes rusty, leading to increased electron leakage and reduced ATP generation~\cite{Green2011}.
In addition to accumulation of damages in the nuclear DNA (see \textbf{Section~\ref{hmark:genomic.instability}}), 
mitochondrial DNA (mtDNA) is also considered to be vulnerable to somatic mutations due to limited repair mechanisms and oxidative microenvironment,
leading to impaired functionality of mitochondria.
Moreover, an age-related increase in reactive oxygen species (ROS) was suggested to cause global cellular damage after a certain threshold, 
further contributing to the emergence of aging-associated phenotypes~\cite{Hekimi2011}.

\subsubsection{Cellular senescence}~\label{hmark:cell.senes}
Cellular senescence can be defined as permanent arrest of the cell cycle.
It can be triggered by a number of factors including, telomere shortening, DNA damage and a number of mitogenic alterations.
The number of senescent cells was shown to increase with age.
Although cellular senescence is originally a protective mechanism preventing the proliferation of damaged cells,
their accumulation during aging results in deleterious effects on tissue homeostasis, further contributing to aging~\cite{Lopez2013}.

\subsubsection{Stem cell exhaustion}
It is long known that aging is accompanied by a decline in stem cell numbers and renewal capacity, 
contributing to the declined homeostatic and regenerative capacity of aged tissues~\cite{Oh2014}.
The factors contributing to stem cell exhaustion includes
DNA damage (\textbf{Section~\ref{hmark:genomic.instability}}), 
epigenetic alterations (\textbf{Section~\ref{hmark:epigenetic}}), 
aggregaion of damaged proteins (\textbf{Section~\ref{hmark:proteosis}}),
accumulation of toxic metabolites (i.e., ROS) and
mithochondrial dysfunction (\textbf{Section~\ref{hmark:mt.dysfunction}}).
Moreover, experimental rejuvenation of stem cells was found to reset the aging clock, 
meaning that it has potential to reverse aging-associated phenotypes~\cite{Rando2012}.

\subsubsection{Altered intercellular communication}~\label{hmark:altered.comm}
In addition to cell-autonomous hallmarks, the last hallmark of aging is related to altered communication of cells 
in terms of endocrine, neuroendocrine and neuronal signaling~\cite{Russell2007}.
A harmonious intercellular communication was suggested to be a key factor for stress response, cell survival and maintaining homeostasis~\cite{Tan2021}.
The aging period is associated with increased inflammatory reactions, decreased immunosurveillance and a changed extracellular environment,
all of which contribute to the deregulation of neurohormonal signaling.
Specifically, senescent cells were known to show an altered secretome, which is rich in proinflammatory cytokines, 
which in turn contributes to the emergence of aging-associated phenotypes~\cite{Childs2016, Kuilman2010}.

\section{Age-related gene expression changes}
As high-throughput technologies become more affordable and widely accessible, 
the past two decades have seen a dramatic increase in studies that focus on gene expression changes in the brain during the aging period.
One of the earlier studies conducted by Lu and colleagues found that the expression levels of the genes 
that play important role in synaptic plasticity and mitochondrial were tend to decrease in aging~\cite{Lu2004}. 
They further demonstrated that this decrease is also accompanied by increased promotor damage, 
suggesting that DNA damage may reduce the expression of genes involved in neuronal functioning, 
possibly contributing to the emergence of aging-associated pathologies.

A 2008 study, employing a microarray-based approach, found that the majority of genes in the aging human brain tend to decrease in expression~\cite{Berchtold2008}.
However, they also identified a set of up-regulated genes during aging, which was found to be related to immune activation and inflammation.
These results were also confirmed by an independent study, where they found that, although the majority of genes were down-regulated during aging,
the genes involved in immune and inflammatory responses were found to increase in expression~\cite{Lu2004}.
Combined, dysregulation of immune system genes might be a characteristic of the aging human brain~\cite{Frenk2018}, 
and it may be associated with aging-related phenotypes in the human brain.

Another common gene expression signature of aging is the downregulation of genes encoding mitochondrial ribosomal proteins and components of the electron transport chain.
In their 2013 study, Kumar \textit{et al.} analyzed microarray data generated from the frontal lobe of the cerebral cortex and cerebellum,
and found that genes encoding mitochondrial components tend to be downregulated during aging~\cite{Kumar2013}.
Similar trends were also observed in other model organisms including rodents, flies and worms, and across different tissues from skin to muscle~\cite{Frenk2018},
suggesting that the downregulation of genes encoding mitochondrial proteins may be a characteristic of aging, 
and may contribute to the age-related mitochondrial dysfunction (\textbf{Section~\ref{hmark:mt.dysfunction}}).

Other studies investigated the relationship between the age-related gene expression change patterns during development and aging.
Somel \textit{et al.} analyzed gene and miRNA expression trajectories and found that 
the majority of expression changes observed in the aging period represent reversals or extensions of developmental patterns~\cite{Somel2010}.
Colantuoni \textit{et al.} also found a similar effect in human brain during development~\cite{Colantuoni2011}.
In a more recent study, the authors performed a meta-analysis on gene expression reversals 
and identified a set of genes showing the up-down pattern (i.e., increase during development, decrease during aging).
They further demonstrated that these genes are involved in neuronal and synaptic functions, suggesting that
decreased expression levels during aging may be associated with the stochastic nature of aging~\cite{Donertas2017}.

There are also more recent studies that investigated age-related gene expression change patterns using RNA-Sequencing based approaches.
Yang \textit{et al.}, for example, analyzed RNA-Sequencing data across different human tissues and identified a set of age-associated genes~\cite{Yang2015}.
They further showed that aging-associated genes showing down-regulation were associated with mitochondrial function,
whereas up-regulated age-associated genes were found to be associated with cell death and inflammation reponse~\cite{Yang2015}.
Another RNA-Seq based study also found that the genes that are down-regulated during aging were associated with
neuronal development and the transmission of nerve impulses~\cite{Naumova2012}.
Thus, these studies demonstrate that independent RNA-Seq based approaches also confirm the previous results obtained from microarray-based studies.

\subsection{Age-related heterogeneity changes}~\label{intro:het.change}
It was previously suggested that the aging period is also associated with dysregulation of gene expression and mRNA processing~\cite{Frenk2018},
suggesting a possible increase in age-related gene expression heterogeneity (also called noise or variation) between individuals.
A number of studies were published reporting increased variation between individuals and cells during aging.

In one of the earlier studies investigating age-related heterogeneity change, 
Somel \textit{et al.} demonstrated an increase in age-related gene expression variation during aging,
which was suggested to be a result of the accumulation of stochastic errors during aging~\cite{Somel2006}.
However, they failed to show the functional consequences of increased heterogeneity 
as they could not identify a set of genes that become more heterogeneous with age.
Consistently, another 2006 study also reported an increase in cell-to-cell variation in mouse heart during the aging period~\cite{Bahar2006}.
In a more recent study conducted by Kedlian \& Donertas \textit{et al.}, the authors investigated 
microarray data from the human prefrontal cortex and revealed that increased heterogeneity is a weak but consistent pattern,
which is also associated with a wide range of pathways~\cite{Kedlian2019}.

Some more recent studies leveraged the availability of more advanced technology, single-cell RNA sequencing,
which provides a resolution that is unattainable by previous sequencing methodologies.
In a 2017 study, Martinez-Jimenez \textit{et al.} analyzed single-cell sequencing data of T cells of mice
and found that the cell-to-cell heterogeneity increases during aging among immune cells~\cite{Martinez2017}.
Another study analyzed gene expression data from human pancreas generated by single-cell sequencing technologies,
which also allows the detection of age-related stochastic errors.
They found that aging is associated with a gradual accumulation of stochastic errors,
which leads to increased cell-to-cell heterogeneity during aging~\cite{Enge2017}.
Similar results were also observed in mice lung by Angelidis \textit{et al.} 
where they found an increased transcriptional heterogeneity during aging~\cite{Angelidis2019}.

Increased heterogeneity during aging was not only observed at the transcriptome level.
A 2002 that analyzed cellular changes during aging in \textit{C. elegans} using the electron-microscopic data also found increased variance in age-related cellular decline.
The authors further discussed that the observed effect might be a result of weakened gene regulation in post-reproductive life, 
suggesting the significant role of stochastic factors~\cite{Herndon2002}.
Another study analyzed age-related heterogeneity changes in monozygotic twins using both transcriptomic and epigenetic datasets.
They found that the heterogeneity in the epigenetic modification patterns between monozygotic twins increases with age.
Moreover, by analyzing gene expression patterns, they demonstrated that 50 years old twins display significantly different gene expression profiles,
whereas 3 years old twins have almost identical expression profiles~\cite{Fraga2005}.

Given all the above, it should also be of note that the generalizability of this effect still remains unclear.
For example, analysis of single-cell profiles of aging mouse brain suggested that 
aging may not be broadly associated with increased transcriptional heterogeneity~\cite{Ximerakis2019}.
Moreover, Vinuela \textit{et al.} \textit{et al.} analyzed age-dependent heterogeneity changes in a twin cohort
and observed a decrease in gene expression heterogeneity for the majority of genes~\cite{Vinuela2018}.

\section{Research Objectives}
Traditionally, molecular aging studies have studied senescence as a monomorphic process. 
In recent years, however, a number of studies have gone beyond this approach, 
and have been reporting changes in gene expression heterogeneity with age. 
Still, the generality of increase in heterogeneity with age remains contentious.
Moreover, whether age-related heterogeneity change is a function of time that starts early in development or 
is limited to the aging period has not been systematically explored.
Also, the functional role of increased heterogeneity and 
its potential contribution to the emergence of aging-associated phenotypes in the human brain still remains unclear.

In this study, I aimed to address these standing questions by analyzing 19 microarray datasets from 3 independent studies covering diverse human brain regions.
The previous research mainly focused on significant changes in individual datasets, which is sensitive to sample size and highly affected by confounding factors.
Thus, in this study, I adapted a meta-analysis approach to analyze consistent changes in gene expression heterogeneity across multiple datasets.
Using this approach, I managed to reduce the effects of confounding factors and technical noise, and identify weak but consistent patterns across datasets.
