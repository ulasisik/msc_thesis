% CHAPTER 1
\chapter{Introduction}
\label{chp:b1}

\section{Aging}
Aging can be defined as a time-dependent deterioration of multiple biological functions and processes.
Decreasing the capacity of an organism to maintain homeostasis, 
aging is associated with an increased vulnerability to many diseases including cancer, cardiovascular and neurodegenerative disorders~\cite{Niccoli2012}.

Aging is accompanied by the accumulation of damage at all levels of an organism, from simple molecules to organs.
It is well known that replication, transcription and translation are error prone processes.
Furthermore, the protective functions against these damages are also known to be vulnerable and tend to produce other forms of damages~\cite{Gladyshev2016}.
These damages are exacerbated during the aging period and further contribute to aging and age-associated diseases.
While both stochastic and deterministic factors contribute to this damage, 
the deterministic (i.e., genetic) component is considered to be more important for distantly related species.
The effect of the genetic component, for example, can be observed in the comparison of lifespans of a human and a mouse living in the same environment.
The stochastic component, on the other hand, are suggested to be an important factor in explaning the differences within homogenous populations~\cite{Gladyshev2016}.
The stochastic factors account for the damage to nucleic acids (i.e., DNA), proteins, lipids and metaboites, as well as other age-related deleterious changes.

\subsection{Start of aging period}
Although, in everyday life, the age of 65 years old is often used to denote the beginning of the old age, the reason of choosing this age is actually historical, rather than being biological~\cite{Covey1992}.
In the biological context, however, the age-related changes were considered to manifest themselves after the organism reaches its maximum reproductive capacity~\cite{Vijg2009}.
In human populations, the age of 20 years is approximately correspond to the age of reproductive maturity~\cite{Walker2006}.

Moreover, earlier studies were also showed that the aging-associated structural changes in human brain begin to exhibit themselves when the individuals in their 20s.
These changes include declines in regional brain volumes~\cite{Sowell2003}, myelin integrity~\cite{Sullivan2006}, cortical thickness~\cite{Salat2004, Magnotta1999},
serotonin receptor binding~\cite{Sheline2002}, and concentration of several brain metabolitees~\cite{Salthouse2009, Kadota2001}.
Recent studies analyzing transcriptional patterns in human brain further revealed that the age of 20 years is a turning point in age-related gene expression trajectories,
suggesting that it roughly corresponds to point at which the aging-associated changes started.

\subsection{Hallmarks of aging}
In their 2013 paper, López \textit{et al.} reviewed 9 hallmarks of aging that were suggested to be the contributors of aging~\cite{Lopez2013}.
The authrors used three criteria while considering candidate hallmarks. 
First, a hallmark must exhibit itself during normal aging.
Second, experimental worsening of a hallmark must accelerate aging.
Third, experimental enhancement of a hallmark must slow down the aging process. 

Moreover, the hallmarks were further divided into 3 categories:
\begin{enumerate}
    \item \textbf{Primary hallmarks}, including genomic instability, telomere attrition, epigenetic alterations and loss of proteostasis,
    were considered to be primary causes of damage.
    \item \textbf{Antagonistic hallmarks}, including deregulated nutrient sensing, mitochondrial dysfunction and cellular senescence,
    were promoted (or accelerated) by the primary hallmarks, and they contribute to further accumulation of damages.
    \item \textbf{Integrative hallmarks}, including stem cell exhaustion and altered intercellular communication, 
    were suggested to arise due to accumulation of damages caused by the primary and antagonistic hallmarks,
    and they have effect on tissue homeostasis and function.
\end{enumerate}

\begin{figure}[h]
    \centering
    \includegraphics[width=.9\textwidth]{figures/figure1_1.jpeg}
    \caption{The hallmarks of aging~\cite{Lopez2013}.
    }~\label{fig:fig1.1}
\end{figure}

Each hallmark of aging is explained in detail in the following subsections.
\subsubsection{Genomic instability}~\label{hmark:genomic.instability}
Genomic instability was suggested to be one of the major stochastic mechanism of aging. 
Many studies previously demonstrated the accumulation of somatic mutations during aging in human and other model organisms~\cite{Moskalev2013,Lodato2018,Lombard2005,Vijg2004,Lu2004}.
The cumulative effect of somatic mutations further disturb the normal functioning of essential genes and transcriptinal pathways, contributing to aging and aging-associated disseases.
Moreover, deficincies in the DNA repair mechanism were found to be associated with accelarated aging, whereas experimental reinforcement of repair mechanism resulted in delayed aging.

\subsubsection{Telomere attrition}
Telomeres are repetitive and protective DNA sequences found at the end of the chromosomes.
Since most mammalian cells do not express telomerase, an enzyme responsible for maintaining telomere length, 
telomeres tend to shorten as the cell divide (i.e., as the organism age). 
Therefore, telomere length limits the proliferation capacity of somatic cells that do not express telomerase enzyme~\cite{Blasco2007}.
Moreover, telomere dysfunction was shown to be associated with accelarated aging~\cite{Armanios2009}, 
whereas experimental induction of telomerase suggested to lead to delayed aging~\cite{Tomas2008}.

\subsubsection{Epigenetic alterations}~\label{hmark:epigenetic}
A number of epigenetic alterations including alterations in DNA methlation patterns, chromatin remdeling and post-transcriptional modifications of histones, 
were suggested to constitute aging-associated epigenetic marks, significantly affecting the normal functioning of cells.
The most notable affect of age-related epigenetic alterations is on transcriptinal outcomes, given the key role of epigenetic factors in transcriptinal regulation.
It was suggested that epigenetic alterations may cause abnormal production and maturation of some mRNAs, and even further leading to increased transcriptional variation~\cite{Lopez2013}.

\subsubsection{Loss of proteostasis}~\label{hmark:proteosis}
Proteostasis, mechanisms that involve in preserving the stabilitty and functionality of proteome, is suggested to be altered during aging~\cite{Koga2011}. 
Age-associated impairment of proteostasis leads to continouos expression of misfolded and aggregated proteins, 
whose accumulation during aging further contributes to the development of age-associated diseases, including Alzheirmer's disease and Parkinson's disease~\cite{Powers2009}.

\subsubsection{Deregulated nutrient sensing}
Nutrient-sensing pathways play important role in sensing the presence or absence of extra- annd intra-cellular nutrients, and they further regulates their intake.
The Insulin/Insuline-like growth factor signaling (IIS) pathway is one of the nutrient-sensing pathways that was found to be regulating aging.
The downstream intracellular effectors of the IIS pathway include AKT, mTOR and FoxO, all of which were suggested to be associated with aging~\cite{Fontana2010, Barzilai2012, Kenyon2010}.
A decreased activity of the IIS and mTOR signaling pathways, for example, was found to extend life span in many model organisms~\cite{Fontana2010}. 

\subsubsection{Mitochondrial dysfunction}~\label{hmark:mt.dysfunction}
During the normal aging, the mitochondrial machinery becomes rusty, leading to increased electron leakage and reduced ATP generation~\cite{Green2011}.
In addition to accumulation of damages in the nuclear DNA (see \textbf{Section~\ref{hmark:genomic.instability}}), 
mitochondrial DNA (mtDNA) is also considered to be vulnerable to the somatic mutations due to limited repair mechanisms and oxidative microenvironment,
leading to impaired functionality of mitochondria.
Moreover, age-related increase in reactive oxygen species (ROS) was suggested to cause global cellular damage after a certain treshold, 
further contributing to the emergence of aging-associated phenotypes~\cite{Hekimi2011}.

\subsubsection{Cellular senescence}~\label{hmark:cell.senes}
Cellular senescence can be defined as permanent arrest of the cell cycle.
It can be triggered by a number of factors including, telomore shortening, DNA damage and a number of mitogenic alterations.
Number of senescent cells were shown to increase with age.
Although cellular senescence is originally a protective mechanism preventing the proliferation of damaged cells,
their accumulation during aging results in deleterious effects on tissue homeostasis, further contributing to aging~\cite{Lopez2013}.

\subsubsection{Stem cell exhaustion}
It is long known that aging is accompanied by a decline in stem cell numbers and renewal capacity, 
contributing to declined homeostatic and regenerative capacity of aged tissues~\cite{Oh2014}.
The factors contributing to stem cell exhaustion includes
DNA damage (\textbf{Section~\ref{hmark:genomic.instability}}), 
epigenetic alterations (\textbf{Section~\ref{hmark:epigenetic}}), 
aggregaion of damaged proteins (\textbf{Section~\ref{hmark:proteosis}}),
accumulation of toxic metabolites (i.e., ROS) and
mithochondrial dysfunction (\textbf{Section~\ref{hmark:mt.dysfunction}}).
Moreover, experimental rejuvenation of stem cells was found to reset the aging clock, 
meaning that it has potential to reverse aging-associated phenotypes~\cite{Rando2012}.

\subsubsection{Altered intercellular communication}
In addition to cell-autonomous hallmarks, the last hallmark of aging is related to altered communication of cells 
in terms of endocrine, neuroendocrine and neuronal signaling~\cite{Russell2007}.
A harmonious intercellular communication was suggested to be a key factor for stress response, cell survival and maintaining homeostasis~\cite{Tan2021}.
The aging period is associated with increased inflammatory reactions, decreased immunosyrveillance and changed extracelular environment,
all of which contribute to the deregulation of neurohormonal signaling.
Specifically, senescent cells were known to show an altered secretome, which is rich in proinflammatory cytokines, 
which in turn contributes to the emmergence of aging-associated phenotypes~\cite{Childs2016, Kuilman2010}.

\section{Age-related gene expression changes}
As high-throughput technologies become more affordable and widely accessible, 
the past two decades have seen a dramatic increase in studies that focus on gene expression changes in brain during the aging period.
One of the earlier studies conducted by Lu and colleagues found that the expression levels of the genes 
that play important role in synaptic plasticity and mitochondrial were tend to decrease in aging~\cite{Lu2004}. 
They further demonstrated that this decrease is also accompanied by increased promotor damage, 
suggesting that DNA damage may reduce the expression of genes involved in neuronal functioning, 
possibly contributing to the emergence of aging-associated pathologies.

One common gene expression signature of aging is the downregulation of genes encoding mithochondrial ribosomal proteins and components of the electron transport chain.
In their 2013 study, Kumar \textit{et al.} analyzed microarray data generated from the frontal lobe of the cerebral cortex and cerebellum,
and found that genes encoding mitochondrial components tend to be downregulated during aging~\cite{Kumar2013}.
Similar trends were also observed in other model organisms including rodents, flies and worms, and across different tissues from skin to muscle~\cite{Frenk2018},
suggesting that the downregulation of genes encoding mithochondrial proteins may be a characteristic of aging, 
and may contribute to the age-related mitochondrial dysfunction (\textbf{Section~\ref{hmark:mt.dysfunction}}).




\subsection{Age-related heterogeneity changes}
It is known that the aging period is also associated with dysregulation of gene expression and mRNA processing.
There have been 


\section{Research Objectives}
Research objectives



