% CHAPTER 1
\chapter{Introduction}
\label{chp:b1}

\section{Aging}
Aging can be defined as a time-dependent deterioration of multiple biological functions and processes.
Decreasing the capacity of an organism to maintain homeostasis, 
aging is associated with an increased vulnerability to many diseases including cancer, cardiovascular and neurodegenerative disorders~\cite{Niccoli2012}.

Aging is accompanied by the accumulation of damage at all levels of an organism, from simple molecules to organs.
It is well known that replication, transcription and translation are error prone processes.
Furthermore, the protective functions against these damages are also known to be vulnerable and tend to produce other forms of damages~\cite{Gladyshev2016}.
These damages are exacerbated during the aging period and further contribute to aging and age-associated diseases.
While both stochastic and deterministic factors contribute to this damage, 
the deterministic (i.e., genetic) component is considered to be more important for distantly related species.
The effect of the genetic component, for example, can be observed in the comparison of lifespans of a human and a mouse living in the same environment.
The stochastic component, on the other hand, are suggested to be an important factor in explaning the differences within homogenous populations~\cite{Gladyshev2016}.
The stochastic factors account for the damage to nucleic acids (i.e., DNA), proteins, lipids and metaboites, as well as other age-related deleterious changes.

\subsection{Start of aging period}
Although, in everyday life, the age of 65 years old is often used to denote the beginning of the old age, the reason of choosing this age is actually historical, rather than being biological~\cite{Covey1992}.
In the biological context, however, the age-related changes were considered to manifest themselves after the organism reaches its maximum reproductive capacity~\cite{Vijg2009}.
In human populations, the age of 20 years is approximately correspond to the age of reproductive maturity~\cite{Walker2006}.

Moreover, earlier studies were also showed that the aging-associated changes in human brain begin to exhibit themselves when the individuals in their 20s.
These changes include declines in regional brain volumes~\cite{Sowell2003}, myelin integrity~\cite{Sullivan2006}, cortical thickness~\cite{Salat2004, Magnotta1999},
serotonin receptor binding~\cite{Sheline2002}, and concentration of several brain metabolitees~\cite{Salthouse2009, Kadota2001}.
Recent studies analyzing transcriptional patterns in human brain further revealed that the age of 20 years is a turning point in age-related gene expression trajectories,
suggesting that it roughly corresponds to point at which the aging-associated changes started.

\subsection{Hallmarks of aging}
In their 2013 paper, López \textit{et al.} reviewed 9 hallmarks of aging that were suggested to contributor of aging~\cite{Lopez2013}.
The authrors used three criteria while considering candidate hallmarks. 
First, a hallmark must exhibit itself during normal aging.
Second, experimental worsening of a hallmark must accelerate aging.
Third, experimental enhancement of a hallmark must slow down the aging process. 

Moreover, the hallmarks were further divided into 3 categories:
\begin{enumerate}
    \item \textbf{Primary hallmarks}, including genomic instability, telomere attrition, epigenetic alterations and loss of proteostasis,
    were considered to be primary causes of damage.
    \item \textbf{Antagonistic hallmarks}, including deregulated nutrient sensing, mitochondrial dysfunction and cellular senescence,
    were promoted (or accelerated) by the primary hallmarks, and they contribute to further accumulation of damages.
    \item \textbf{Integrative hallmarks}, including stem cell exhaustion and altered intercellular communication, 
    were suggested to arise due to accumulation of damages caused by the primary and antagonistic hallmarks,
    and they have effect on tissue homeostasis and function.
\end{enumerate}

\begin{figure}[h]
    \centering
    \includegraphics[width=.9\textwidth]{figures/figure1_1.jpeg}
    \caption{The hallmarks of aging~\cite{Lopez2013}.
    }~\label{fig:fig1.1}
\end{figure}

Each hallmark of aging is explained in detail in the following subsections.
\subsubsection{Genomic instability}
\subsubsection{Telomere attrition}
\subsubsection{Epigenetic alterations}
\subsubsection{Loss of proteostasis}
\subsubsection{Deregulated nutrient sensing}
\subsubsection{Mitochondrial dysfunction}
\subsubsection{Cellular senescence}
\subsubsection{Stem cell exhaustion}
\subsubsection{Altered intercellular communication}

\section{Age-Related Changes in human brain}
Age-Related Changes in Human Brain
\subsection{Structural Changes}
\subsection{Transcriptome Changes}

\section{Intraindividual variability in aging}
phenotypic (behavior, cognitive tasks, vs..) variance
increased variance between individuals during cognitive aging
Gene Expression variance

\section{Research Objectives}
Research objectives



