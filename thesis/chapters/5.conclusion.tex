% CHAPTER 5
\chapter{Conclusion}
\label{chp:b5}

Aging is a complex process characterized by a gradual functional decline.
Moreover, aging is considered to pose a major risk factor for many diseases, including cancer, and cardiovascular and neurodegenerative disorders.
Transcriptome studies focusing on age-related changes in the human brain have been offering novel insights into the understanding of underlying mechanisms.

In this study, I conducted a meta-analysis of 19 microarray datasets containing 1,010 samples from 17 brain regions and covering the whole lifespan.
Specifically, I investigated the changes in inter-individual gene expression heterogeneity during aging in comparison to the development period.
The main findings of this study were summarized in the following bullet points below.

\begin{itemize}
    \item There are more genes showing significant changes in gene expression during development compared to aging.
    \item In development, the majority of genes showing significant expression change decrease in expression.
    \item Development datasets show higher coordination in gene expression changes, compared to aging datasets.
    \item Gene expression heterogeneity consistently increases with age during aging (20 to 98 years of age) but not in postnatal development (0 to 20 years of age).
    \item The heterogeneity increase observed in aging comes with physiological consequences, 
    such that the genes showing consistent effects are associated with biological processes important for life- and health-span regulation 
    (e.g. autophagy, mTOR signaling), as well as for cognitive functions (e.g. axon guidance, postsynaptic specialization).
    \item Not only specific regulators (miRNAs and transcription factors) but also the number of regulators is positively associated with consistent changes in heterogeneity. 
\end{itemize}
