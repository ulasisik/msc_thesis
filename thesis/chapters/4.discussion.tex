% CHAPTER 4
\chapter{Discussion}
\label{chp:b4}

In this study, I aimed to investivage the changes in gene expression heterogeneity during development and aging periods.
The dataset that I analyzed included 19 microarray dataset containing gene expression measurements for human brain from 3 independent sources.
Overall, 1,010 samples from 17 different brain regions of 298 individuals (\textbf{Table~\ref{table:table1}}, \textbf{Figure~\ref{fig:fig2.1}}).

The datasets, containing samples covering whole lifespan (ages from 0 to 98 years old), 
were first divided into development and aging datasets,
using the age of 20 years as a seperation point (see \textbf{Section~\ref{subsec:dset.seperation}}), 
which was previously shown to be the global turning point of gene expression trajectories~\cite{Donertas2017, Colantuoni2011, Somel2010}.
Overall, I obtained 19 development datasets including samples whose ages range from 0 to 20 years old (n = 441).
It is also important to note that pre-natal samples were excluded from the downstream analysis
since the gene expression trajectories are suggested to be discontinous between pre- and post-natal development, 
and the scope of this study was to compare heterogeneity changes during postnatal development and aging.
The aging datasets (n = 19 datasets), on the other hand, included samples whose ages range from 20 to 98 (n = 569).
Only the common genes (i.e., the genes for which I have measurement across all datasets) were included in the downstream analysis (n = 11,137).

Using the advantage of having multiple datasets, this study focused on consistent changes that are shared across the datasets, 
rather than focusing on significant changes in a single dataset, which is highly dependent on the sample sizes. 
Thus, this approach was able to capture weak but shared signals that would otherwise fail to pass the significance treshold in individual datasets.

\section{Correlations among datasets in expression and heterogeneity changes}
After performing preprocessing on microarray datasets (see \textbf{Section~\ref{sec:dset.preprocess}}), 
I first sought to characterize the age-related changes in gene expression, 
by performing a linear regression analysis between scaled expression values and 
fourth root of ages in days as shown in the left panels of \textbf{Figure~\ref{fig:fig2.3}}.
The $\beta_{i1}$ values obtained from \textbf{Equation~\ref{eq:exp_change}} were considered as the measure of age-related expression change.
The regression analysis was performed for each gene and for each time period, seperately (see \textbf{Section~\ref{sec:exp-change}} for details).

Then I investigated the coordination in expression change between all possible pairs of datasets by calculating Spearman's correlation coefficient 
and found that the correlation among development datasets (median correlation coefficient = 0.56) 
is significantly higher than the correlation among aging datatsets (median correlation coefficient = 0.43, permutation test p-value = 0.003).
Furthermore, more genes showed significant changes during development compared to aging period, and they mostly tended to decrease in expression (\textbf{Figure~\ref{fig:fig3.1}}).
One possible explanation of these results might be related to stochastic nature of aging. 
As previously suggested, the accumulation of random detrimental effects (i.e., mutations) during aging may cause reduced gene expressions, 
and in turn causing an increased level of heterogenety in aging~\cite{Lu2004}.
Consistent with earlier findings, my initial analysis of gene expression changes also supports the view of aging as a stochastic process.

Next, the change in gene expression heterogeneity was characterized 
by performing Spearman's correlation test between absolute value of residuals obtained from \textbf{Equation~\ref{eq:exp_change}} 
and the fourth root of age for each gene and time period seperately (see \textbf{Section~\ref{sec:het-change}}).
Analyzing the correlations among development and aging datasets in heterogeneity change,
I found that aging datasets display a higher correlation compared to development datasets, 
reflecting a more consistent heterogeneity change in aging.
A further analysis of genes showing significant heterogeneity changes revealed that 
there are more genes showing significant changes in heterogeneity during aging, compared to development (\textbf{Figure~\ref{fig:fig3.2}b}). 
Moreover, the significant changes in heterogeneity are mostly in the positive direction during aging, suggesting an increase in heterogeneity.

\section{Increased heterogeneity consistency in aging}
Having observed more consistent heterogeneity change and more significant heterogeneity increase in aging, 
I next investigated heterogeneity changes in individual datasets and found an overall increase in heterogeneity during aging 
(i.e., 18 of 19 aging datasets display higher median heterogeneity change compared to development, see \textbf{Figure~\ref{fig:fig3.3}a}). 
An analysis of consistent heterogeneity change further revealed that there is a significant shift towards increased heterogeneity consistency during aging compared to expectation,
while no such shift was observed for development datasets (\textbf{Figure~\ref{fig:fig3.3}c}).

There are number of factors that can explain increased heterogeneity during aging compared to development.
First, a number of studies previously demonstrated that the stochastic accumulation of somatic mutations may cause genomic instability,
which may in turn lead to increased heterogeneity in aging period~\cite{Lu2004, Vijg2004, Lodato2018, Lombard2005}.
Second, Cheung \textit{et al.}, analyzing a twin cohort, demonstrated the stochastic nature of age-related changes in chromatin,
leading to increase variation between both individuals and cells in aging~\cite{Cheung2018}.
The third factor might be the transcriptional regulation, which was suggested to be stochastic due to randomness of biochemical reactions~\cite{Maheshri2007, Barroso2018}.
Previous studies found that the variability in gene expression is positively correlated with the number of transcription factors that control is regulation~\cite{Barroso2018, Sharon2014}.
While the first two factors could not be tested in this study since the datasets did not contain somatic mutation or epigenetic regulation information, 
I also found a mainly positive correlation between number of regulators and heterogeneity change during aging (\textbf{Figure~\ref{fig:fig3.7}}).
Moreover, transcription factors found to be significantly associated with increased heterogeneity during aging included FoxO and EGR family of transcription factors, 
which were shown to be regulating genes important for synaptic homeostasis, stress resistance, cell cycle arrest and apoptosis.
Combined, the results obtained in this study also supported the view that increased heterogeneity in aging may be associated with the transcriptional regulation.

\section{Increased heterogeneity may have important functional consequences}
functional enrichment analysis
transcriptional regulator anaysis

\section{Increased heterogeneity is a biological signal}
I next confirmed that the observed increase in heterogeneity consistency is a biological signal, rather than being a technical artifacts or a result of low statistical power, 
given the similar sample sizes of development and aging periods (\textbf{Figure~\ref{fig:fig2.1}}).

One technical factor that can explain increased heterogeneity during aging might be related to dependence between mean and variance, 
where accompanying increase in expression levels may cause increased variance, which in turn detected as increased heterogeneity.
To address this issue, I performed correlation test between heterogeneity changes and expression changes for each dataset and each period, 
and found that the correlation coefficients calculated for aging datasets are mostly negative (\textbf{Figure~\ref{fig:fig3.3}b}), 
suggesting that observed increase in heterogeneity was not caused by mean-variance dependence.

Another technical factor that can cause increased inter-individual variance may be related to post-mortem interval (PMI), 
which measures the time between death and sample collection.
It was previously suggested that PMI-related mRNA degradation is gene-specific, leading to a bias in downstream analysis~\cite{Zhu2017}.
To confirm that increased heterogeneity was not a result of PMI-related mRNA degradation,
I used previously identified 107 PMI-associated genes~\cite{Zhu2017}, 75 of which were included in this analysis.
Specifically, I tested if the 75 PMI-associated genes show more increase in heterogeneity during aging,
and found that only 2 of 147 consistent genes were PMI-associated, 
suggesting that PMI by itself was not enough to explain observed increase in heterogeneity (\textbf{Figure~\ref{fig:a6.1}}).

One other factor that can affect the main results presented here might be related to age scales.
As previously stated, the fourth root of age scale was used in this study to obtain relatively uniform distribution of ages across the lifespan.
However, whether the observed increase in heterogeneity depends on the use of specific age scales remained unanswered.
To assess the effect of using different age scales on the downstream analysis, I repeated the analysis using 3 additional age scales: 
(1) age in days, (2) age in log2 scale, and (3) age in years (\textbf{Figure~\ref{fig:a6.2}}).
Overall, I found that using different age scales also yield a quantitatively similar results.
In fact, the use of log2 age scale resulted in a higher number of genes showing consistent increase in heterogeneity across all 19 datasets (\textbf{Figure~\ref{fig:a6.2}b}, lower left panel).
Nevertheless, this analysis indicated that the observed increase in heterogeneity is not a result of the use of specific age scale.

It was previously suggested a sex-specific difference in human brain aging, where males showed more changes in gene expression~\cite{Berchtold2008}.
In this analysis, however, both males and females were combined to calculate expression and heterogeneity changes, raising a question about possible confound of sex with age.
To address this question, I retrieved the real values of residuals obtained from \textbf{Equation~\ref{eq:exp_change}} 
(not absolute values) for 147 genes showing consistent increase in heterogeneity.
Then, for each gene and each aging dataset, two sample Wilcoxon test was performed on residuals to test if there is a significant difference between males and females.
The obtained p-values were corrected for multiple testing by B{\&}H method (\textbf{Section~\ref{subsec:p.adjust}}).
Overall, I found that there are only 15 out of 147 consistent genes show significant difference between sexes in at least one dataset (\textbf{Figure~\ref{fig:a6.3}}),
suggesting that the increased heterogeneity cannot be explained solely by sex-specific differences in brain aging.

In this study, a permutation scheme that takes into account the dependency of Kang2011 and Somel2011 
datasets was employed to calculate expectation of heterogeneity consistency (\textbf{Section~\ref{sec:perm}}).
By also using random permutations to calculate expected consistency in heterogeneity increase, 
I found that the scheme used in this study was more strict than the random permutations (\textbf{Figure~\ref{fig:a6.4}}).

One important assumption of this study is that the relationship between scaled expression levels and fourth root of age is linear,
since linear regression was used to characterize the age-related expression change.
To ensure that this assumption did not have significant effect on the downstream results, 
I re-calculated heterogeneity changes using the residuals obtained from loess regression and 
found a high correlation between heterogeneity changes calculated using linear regression and loess regression (\textbf{Figure~\ref{fig:a6.5}}).
Yet, the heterogeneity changes calculated from loess regression did not included in the downstream analysis
since both the model parameters and sample sizes have significant effect on the estimates of loess regression.

The last factor that can cause the observed increase in heterogeneity might be related to the outliers in the datasets.
For example, one older individual having too low or high expression value (i.e., having higher absolute value of residual) 
can drive the heterogeneity estimates up.
To investigate the effect of outliers, I plotted the absolute value of residuals for 147 consistent genes (\textbf{Figure~\ref{fig:a6.5}}).
A visual inspection revealed that there was no significant outlier sample that can explain the observed increase in heterogeneity.

Overall, these extra analysis demonstrated that the increased heterogeneity reported in this study cannot be explained by low statistical power and technical factors,
but rather it is indeed a biological signal.

\section{Limitations \& future perspectives}
There are a number of limitations of this study that should be pointed out. The first one is related to microarray data. 
- microarray data -> let me to analyze reative changes only
- bulk rna -> change in the cel-type composition?
- Cannot confidently identify gene-sets showing consistent increase.


